% ---
% Dedicatória
% ---
\begin{dedicatoria}
   \vspace*{\fill}
   \centering
   \noindent
   \textit{ Este trabalho é dedicado aos xxxxxxxxxxxx, que assim como eu encerram uma difícil etapa da vida acadêmica. Dedico este trabalho a todo xxxxxxxxxxxxx, corpo docente e discente, a quem fico lisonjeado por dele ter feito parte.} \vspace*{\fill}
\end{dedicatoria}
% ---

% ---
% Agradecimentos
% ---
\begin{agradecimentos}
\lipsum[2]



\end{agradecimentos}
% ---

% ---
% Epígrafe
% ---
\begin{epigrafe}
    \vspace*{\fill}
	\begin{flushright}
		\textit{``Não vos amoldeis às estruturas deste mundo, \\
		mas transformai-vos pela renovação da mente, \\
		a fim de distinguir qual é a vontade de Deus: \\
		o que é bom, o que Lhe é agradável, o que é perfeito.\\
		(Bíblia Sagrada, Romanos 12, 2)}
	\end{flushright}
\end{epigrafe}
% ---

% ---
% RESUMOS
% ---

% resumo em português
\setlength{\absparsep}{18pt} % ajusta o espaçamento dos parágrafos do resumo
\begin{resumo}
\SingleSpacing
%destacar o assunto do trabalho, o objetivo, o método e os resultados esperados. Deve apresentam de 150 a 200 palavras, escritas em formato de bloco,espaço simples e fonte menor (um  ponto) que a do texto do projeto
 \lipsum[1]

 Palavras-chave: \imprimirpalavraschave
\end{resumo}

% resumo em inglês
\begin{resumo}[Abstract]
 \begin{otherlanguage*}{english}
%  \linespread{1.3}
\SingleSpacing
 \lipsum[3]
   \vspace{\onelineskip}
 
   \noindent 
   Keywords: \imprimirkeywords
 \end{otherlanguage*}
\end{resumo}