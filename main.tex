\documentclass[
	% -- opções da classe memoir --
	12pt,				% tamanho da fonte
	openright,			% capítulos começam em pág ímpar (página vazia se preciso)
% 	twoside,			% para impressão em recto e verso.
    oneside,
	a4paper,			% tamanho do papel. 
	% -- opções da classe abntex2 --
	chapter=TITLE,
 	section=TITLE,
% 	subsection=Title,
% 	subsubsection=Title,
	% -- opções do pacote babel --
	english,			% idioma adicional para hifenização
% 	french,				% idioma adicional para hifenização
% 	spanish,			% idioma adicional para hifenização
	brazil				% o último idioma é o principal do documento
	]{abntex2}

% ---
% Pacotes básicos 
% ---
%\usepackage{lmodern}		% Usa a fonte Latin Modern			
\usepackage{helvet}		% Usa a fonte Arial
\renewcommand{\familydefault}{\sfdefault}

\usepackage[T1]{fontenc}	% Selecao de codigos de fonte.
\usepackage[utf8]{inputenc}	% Codificacao do documento 
\usepackage{indentfirst}	% Indenta o primeiro parágrafo de cada seção.
\usepackage{color}			% Controle das cores
\usepackage{graphicx}		% Inclusão de gráficos
\usepackage{microtype} 		% para melhorias de justificação
\usepackage{setspace}       % para espacamento simples no resumo
\usepackage{pdfpages}
% ---

% ---
% Pacotes adicionais, usados apenas no exemplo
% ---
\usepackage{lipsum} % para geração de dummy text
%\usepackage[utf8]{inputenc}
%\usepackage[T1]{fontenc}
%\usepackage[brazil]{babel} 
% ---

% ---
% Pacotes de citações
% ---
%\usepackage[brazilian,hyperpageref]{backref} % Mostra a página onde cada citação foi feita
%\usepackage[num,abnt-etal-list=0,bibjustif]{abntex2cite} % Citações numéricas (ordem de apresentação)
\usepackage[alf,abnt-etal-list=3,bibjustif,abnt-repeated-author-omit=yes,abnt-repeated-author-omit=yes,abnt-url-package=hyperref,abnt-emphasize=bf]{./abntex2cite} % Citações autor-data (ordem alfabética)

% --- 
% CONFIGURAÇÕES DE PACOTES
% --- 

% % ---
% % Configurações do pacote backref, SE FOR USAR DESCOMENTE TODO ESSE TRECHO
% % Usado sem a opção hyperpageref de backref
% \renewcommand{\backrefpagesname}{Citado na(s) página(s):~}
% % Texto padrão antes do número das páginas
% \renewcommand{\backref}{}
% % Define os textos da citação
% \renewcommand*{\backrefalt}[4]{
% 	\ifcase #1 %
% 		Nenhuma citação no texto.%
% 	\or
% 		Citado na página #2.%
% 	\else
% 		Citado #1 vezes nas páginas #2.%
% 	\fi}%
% % ---


% ---
% Informações de dados para CAPA e FOLHA DE ROSTO
% ---
\usepackage{abntex2esao}
\usepackage[T1]{fontenc}
\usepackage[utf8]{inputenc}
\DeclareUnicodeCharacter{200B}{{\hskip 0pt}}
% ---

%------------
%Outro Idioma
%------------
\usepackage{ifthen}
\RequirePackage{ifthen}
\newif\ifen
\newif\ifpt
\newif\ifes
\newif\iffr
\newif\ifita
\newcommand{\en}[1]{\ifen#1\fi}
\newcommand{\pt}[1]{\ifpt#1\fi}
\newcommand{\es}[1]{\ifes#1\fi}
\newcommand{\fr}[1]{\iffr#1\fi}
\newcommand{\ita}[1]{\ifita#1\fi}
\newcommand{\jan}{%
    \en{January}%
    \pt{Janeiro}%
    \es{Enero}%
    \fr{Janvier}%
    \ita{Gennaio}%
}
\pttrue

%\usepackage{caption}
%\usepackage{subcaption}



% ---
% EXEMPLO PARA CURSO DE POS GRADUAÇÃO
% ---

% \instituicao{Instituto Militar de Engenharia}
% \programapgdepartamento{Engenharia de Transportes}
% \nivelestudo{Pós-Graduação Lato Sensu} % Pós-Graduação Lato Sensu ou Mestrado
% % ---
% \titulo{Modelo Canônico de Trabalho Acadêmico com \abnTeX\space \versaoDocumento}
% \palavraschave{signal,signals company,comunicações nas grandes,apoio de comunicações}
% \keywords{unmanned systems,unmanned vehicles,uav,uas,cooperative tasks,intelligent agents}
% % ---
% \autores{Fulano de}{Silva}% 1+ autores
% % \autor{Fulano de}{Tal} %{nome}{sobrenome}
% \orientadores{Sicrano,Beltrano}{Santos,Oliveira}{Ph.D.,D.Sc.}%{nomes}{sobrenomes}{títulos}
% % ---
% \local{Rio de Janeiro}
% \data{2022}
% \datadefesa{09 de setembro de 2022}
% \bancadeexaminadores{
%     Prof. \textbf{Orientador 1} - D.Sc. da EsAO - Presidente,
%     Prof. \textbf{Orientador 2} - D.Sc. da ECEME,
%     Prof. \textbf{Professor 1} - Oficial QEMA da EsAO,
%     Prof. \textbf{Professor 2} - D.Sc. da EsAO,
% }

% ---



% ---
% EXEMPLO PARA PROJETO DE FIM DE CURSO
% ---

\instituicao{Escola de Aperfeiçoamento de Oficiais}
\programapgdepartamento{Ciências Militares}
\nivelestudo{Pós-Graduação Lato Sensu} % Pós-Graduação Lato Sensu ou Mestrado
% ---
\titulo{O Título do Meu Trabalho}
\palavraschave{comando e controle,apoio a decisão,comunicações}
\keywords{command and control,decision support,signals}
% ---
\autores{Fulano}{de Tal}% {nome}{sobrenome} 1+
\orientadores{Ciclano da Silva}{Souza}{Maj Inf}%{nomes}{sobrenomes}{títulos} 1+
% ---
\local{Rio de Janeiro}
\data{2022}
\datadefesa{9 de setembro de 2022}
\bancadeexaminadores{
     FULANO DA SILVA \textbf{SOUZA} - Maj\\ Escola de Aperfeiçoamento de Oficiais\\ Presidente,
     \textbf{BELTRANO} DA SILVA - Maj\\ Escola de Aperfeiçoamento de Oficiais\\ Membro,
     \textbf{CICLANO} SMITH - Cap\\ Escola de Aperfeiçoamento de Oficiais\\ Membro
}

% ---


% ---
% Configurações de aparência do PDF final

% % alterando o aspecto da cor azul
% \definecolor{blue}{RGB}{41,5,195}

% % informações do PDF
\makeatletter
\hypersetup{
    %pagebackref=true,
    % pdftitle={\@title}, 
    % pdfauthor={\@author},
    % pdfsubject={\imprimirpreambulo},
    % pdfcreator={LaTeX with abnTeX2},
    colorlinks=true, % false: boxed links; true: colored links
    linkcolor=black, % color of internal links
    citecolor=black,	% color of links to bibliography
    filecolor=black, % color of file links
    urlcolor=black,
    bookmarksdepth=4
}
\makeatother
% % --- 
% % --- 



% ---
% Posiciona figuras e tabelas no topo da página quando adicionadas sozinhas
% em um página em branco. Ver https://github.com/abntex/abntex2/issues/170
\makeatletter
\setlength{\@fptop}{5pt} % Set distance from top of page to first float
\makeatother
% ---

% ---
% Possibilita criação de Quadros e Lista de quadros.
% Ver https://github.com/abntex/abntex2/issues/176
%
\newcommand{\quadroname}{Quadro}
\newcommand{\listofquadrosname}{Lista de quadros}

\newfloat[chapter]{quadro}{loq}{\quadroname}
\newlistof{listofquadros}{loq}{\listofquadrosname}
\newlistentry{quadro}{loq}{0}

% configurações para atender às regras da ABNT
\setfloatadjustment{quadro}{\centering}
\counterwithout{quadro}{chapter}
\renewcommand{\cftquadroname}{\quadroname\space} 
\renewcommand*{\cftquadroaftersnum}{\hfill--\hfill}

\setfloatlocations{quadro}{hbtp} % Ver https://github.com/abntex/abntex2/issues/176
% ---

% --- 
% Espaçamentos entre linhas e parágrafos 
% --- 

% O tamanho do parágrafo é dado por:
\setlength{\parindent}{1.3cm}

% Controle do espaçamento entre um parágrafo e outro:
\setlength{\parskip}{0.2cm}  % tente também \onelineskip

% ---
% compila o indice
% ---
\makeindex
% ---
%
%    \usepackage{fancyhdr}
%    \pagestyle{fancy}
%    \renewcommand{\headrulewidth}{0pt}
%    \fancyhf{}
%    \cfoot{\noindent\fcolorbox{red}{white}{\parbox{\dimexpr \linewidth-2\fboxsep-2\fboxrule}{%
%                \centering{\color{red}{\textbf{INFORMAÇÃO DE P\&D - ACESSO RESTRITO}\\
%                        §1º do Art. 7º da Lei nº 12.527, de 18 de novembro de 2011\\
%                        Inciso II do Art. 6º do Decreto nº 7.724, de 16 de maio de 2012}}}}}
%

% Marca dagua com o reservado
%\usepackage{draftwatermark} 
%\SetWatermarkColor[rgb]{1,0,0}
%\SetWatermarkAngle{0}
%\SetWatermarkText{\textsc{RESERVADO}}
%\SetWatermarkScale{0.12}
%\SetWatermarkVerCenter{10mm}
%\SetWatermarkHorCenter{150mm}

% acronyms
\usepackage{acro} 
%\acsetup{
%    list/display=used ,
%    list/heading=none
%}
%\DeclareAcronym{xxxxx}{short={xxxxxxxxx},long={xxxxxxxxxxxxx}}
\DeclareAcronym{C²}{short={C²},long={Comando e Controle}}
\DeclareAcronym{END}{short={END},long={Estratégia Nacional de Defesa}}
\DeclareAcronym{EMCFA}{short={EMCFA},long={Estado-Maior Conjunto das Forças Armadas}}
\DeclareAcronym{GU}{short={GU},long={Grande Unidade}}
\DeclareAcronym{SAD}{short={SAD},long={Sistemas de Apoio à Decisão}}
\DeclareAcronym{SI}{short={SI},long={Sistemas de Informação}}
\DeclareAcronym{SISBIN}{short={SISBIN},long={Sistema Brasileiro de Inteligência}}
\DeclareAcronym{SISNACC}{short={SISNACC},long={Sistema Nacional de Comunicações Críticas}}
\DeclareAcronym{TIC}{short={TIC},long={Tecnologia da Informações e Comunicações}}
\DeclareAcronym{SisMC²}{short={SisMC²},long={Sistema Militar de Comando e Controle}}
\DeclareAcronym{CONOPS}{short={CONOPS},long={Conceito Operacional}}
\DeclareAcronym{TI}{short={TI},long={Tecnologia da Informação}}
\DeclareAcronym{MD}{short={MD},long={Ministério da Defesa}}
\DeclareAcronym{OMs}{short={OMs},long={Organizações Militares}}

% ----
% Início do documento
% ----
\begin{document}
    
% Seleciona o idioma do documento (conforme pacotes do babel)
%\selectlanguage{english}
\selectlanguage{brazil}

% Retira espaço extra obsoleto entre as frases.
\frenchspacing 

% ----------------------------------------------------------
% ELEMENTOS PRÉ-TEXTUAIS
% ----------------------------------------------------------
% \pretextual

% ---
% Capa
% ---
\imprimircapa
% ---

% ---
% Folha de rosto
% (o * indica que haverá a ficha bibliográfica)
% ---
\imprimirfolhaderosto*
% ---

% ---
% Inserir a ficha bibliografica
% ---

% \begin{fichacatalografica}
%     \includepdf{fig_ficha_catalografica.pdf}
% \end{fichacatalografica}

\imprimirfichacatalografica
% ---

% ---
% Inserir folha de aprovação
% ---

% \begin{folhadeaprovacao}
% \includepdf{folhadeaprovacao_final.pdf}
% \end{folhadeaprovacao}
%
\imprimirfolhadeaprovacao
% ---

% ---
% Dedicatória
% ---
\begin{dedicatoria}
   \vspace*{\fill}
   \centering
   \noindent
   \textit{ Este trabalho é dedicado aos xxxxxxxxxxxx, que assim como eu encerram uma difícil etapa da vida acadêmica. Dedico este trabalho a todo xxxxxxxxxxxxx, corpo docente e discente, a quem fico lisonjeado por dele ter feito parte.} \vspace*{\fill}
\end{dedicatoria}
% ---

% ---
% Agradecimentos
% ---
\begin{agradecimentos}
\lipsum[2]



\end{agradecimentos}
% ---

% ---
% Epígrafe
% ---
\begin{epigrafe}
    \vspace*{\fill}
	\begin{flushright}
		\textit{``Não vos amoldeis às estruturas deste mundo, \\
		mas transformai-vos pela renovação da mente, \\
		a fim de distinguir qual é a vontade de Deus: \\
		o que é bom, o que Lhe é agradável, o que é perfeito.\\
		(Bíblia Sagrada, Romanos 12, 2)}
	\end{flushright}
\end{epigrafe}
% ---

% ---
% RESUMOS
% ---

% resumo em português
\setlength{\absparsep}{18pt} % ajusta o espaçamento dos parágrafos do resumo
\begin{resumo}
\SingleSpacing
%destacar o assunto do trabalho, o objetivo, o método e os resultados esperados. Deve apresentam de 150 a 200 palavras, escritas em formato de bloco,espaço simples e fonte menor (um  ponto) que a do texto do projeto
 \lipsum[1]

 Palavras-chave: \imprimirpalavraschave
\end{resumo}

% resumo em inglês
\begin{resumo}[Abstract]
 \begin{otherlanguage*}{english}
%  \linespread{1.3}
\SingleSpacing
 \lipsum[3]
   \vspace{\onelineskip}
 
   \noindent 
   Keywords: \imprimirkeywords
 \end{otherlanguage*}
\end{resumo}

% ---
% inserir lista de ilustrações
% ---
\pdfbookmark[0]{\listfigurename}{lof}
\listoffigures*
\cleardoublepage
% ---

% ---
% inserir lista de quadros
% ---
%\pdfbookmark[0]{\listofquadrosname}{loq}
%\listofquadros*
%\cleardoublepage
% ---

% ---
% inserir lista de tabelas
% ---
%\pdfbookmark[0]{\listtablename}{lot}
%\listoftables*
%\cleardoublepage
% ---

% ---
% inserir lista de abreviaturas e siglas
% ---
\begin{siglas}
    \item
    \printacronyms[heading=none]
%    \printacronyms[display=used,heading=none]
\end{siglas}
% ---

% ---
% inserir lista de símbolos
% ---
%\begin{simbolos}
%  \item[$ \Gamma $] Letra grega Gama
%  \item[$ \Lambda $] Lambda
%\end{simbolos}
% ---

% ---
% inserir o sumario
% ---

\makeatletter
\let\oldcontentsline\contentsline
\def\contentsline#1#2{%
    \oldcontentsline{#1}{\MakeTextUppercase{#2}}%
}
\makeatother

\pdfbookmark[0]{\contentsname}{toc}
\tableofcontents*
\cleardoublepage
% ---

% ----------------------------------------------------------
% ELEMENTOS TEXTUAIS
% ----------------------------------------------------------
\textual

% ----------------------------------------------------------
% Introdução (exemplo de capítulo sem numeração, mas presente no Sumário)
% ----------------------------------------------------------
\chapter{Introdução}
% ----------------------------------------------------------

\lipsum[3] \cite{LivroBrancodeDefesaNacional455}.

\lipsum[4] \cite{PoliticaNacionaldeDefesaeEstrategiaNacionaldeDefesa457,LivroBrancodeDefesaNacional455}.

\lipsum[5] \cite{LivroBrancodeDefesaNacional455}.

\lipsum[6] \cite{PoliticaNacionaldeDefesaeEstrategiaNacionaldeDefesa457}.

\lipsum[7] \cite{PoliticaNacionaldeDefesaeEstrategiaNacionaldeDefesa457}.

\lipsum[8-10]

\section{Problema}

\lipsum[11-13] \cite{ConceitoOperacionaldoSistemadeInformacaoedeApoioaDecisaoparaComandoeControle492}.


\section{Objetivos}
\lipsum[14]

\subsection{Objetivo Geral}
\lipsum[15]

\subsection{Objetivos Específicos}
\begin{itemize}
    \item xxxxxxxxxxxxxxxxxx
    \item xxxxxxxxxxxxxxxx \ac{SAD} xxxxxxxxxxxxxxx \ac{SisMC²}; e
    \item xxxxxxxxxxxx \ac{SAD} xxxxxxxxxxxxxxxxxxxxxxx.
\end{itemize}


\chapter{Referencial Teórico}
\noindent
\lipsum[15]

\section{Generalidades}

\lipsum[16]

\subsection{Conceitos da xxxxxxxxxxxxxx}

\begin{citacao}
    \lipsum[17] \cite{DoutrinaMilitarTerrestre54}.
\end{citacao}

Segundo \citeonline{DoutrinaMilitarTerrestre54} xxxxxxxxxxxxxxxxxxxxxxxxx xxxxxxxxxx xxxxxxxxx.

\begin{citacao}
    \lipsum[18] \cite{DoutrinaMilitarTerrestre54}.
\end{citacao}

\lipsum[19]



\section{Principais xxxxxxxxxx em uso na FTer}
\lipsum[20-23] \cite{LivroBrancodeDefesaNacional455,EstruturaMilitardeDefesa454}.


\section{Sistema xxxxxxxx xxxxxxxx}

\lipsum[7]


\section{Aplicações xxxxxxxxxxxxxxxx}

\lipsum[24]

\begin{figure}[!h]
    \centering
    \includegraphics[width=1\linewidth]{img/torchx_dismounted}
    \caption{Datasheet do Torch X dismounted com os principais elementos integradores do sistema.}
    \legend{Fonte: \cite{TorchXDs}}
    \label{fig:torchxdismounted}
\end{figure}

\lipsum[25]

\begin{figure}[!h]
    \centering
    \includegraphics[width=1\linewidth]{img/torchx}
    \caption{Torch X HQ em operação}
    \legend{Fonte: \cite{TorchXHQ}}
    \label{fig:torchxhd}
\end{figure}

\lipsum[26]

\begin{figure}[!h]
    \centering
    \includegraphics[width=1\linewidth]{img/torchx_mounted}
    \caption{Torch X Mounted}
    \legend{Fonte: \cite{TorchX}}
    \label{fig:torchxmounted}
\end{figure}

\lipsum[27]

\section{xxxxxxxx xxxxxxxxxxxxx}

\subsection{Conceitos de xxxxxxxx}

\lipsum[25-30]

\begin{equation}
    \label{math:acuracia}
    AccC_1 = \frac{VP + VN}{n}
\end{equation}


\begin{equation}
    \label{math:abrangencia}
    RevC_1 = \frac{VP}{VP+FN}
\end{equation}



\begin{equation}
    \label{math:precisao}
    PrecC_1 = \frac{VP}{VP+FP}
\end{equation}


\begin{equation}
    \label{math:f1}
    F_1 = \frac{2*Prec*Rev}{Prec + Rev}
\end{equation}

\lipsum[31]

\begin{table}[!ht]
    \centering
    \caption{Matriz de xxxxxxxxx}
    \label{tab:matriz_de_xxxxxx}
    \vspace{0.5cm}
    \begin{tabular}{r|cccc||c}
        & P ($ C_1 $)   & P ($ C_2 $)   & Total Real \\
        \hline
        \hline
        R ($ C_1 $)             & (VP) 120 & (FN) 40  & 160      \\
        R ($ C_2 $)             & (FP) 33  & (VN) 127 & 160      \\
        \hline
        \hline
        Total xxxxxx & 153 & 167 & 320      \\
        \hline
    \end{tabular}
\end{table}

\chapter{Metodologia}

\section{Objeto Formal de Estudo}
\lipsum[32]

\section{Delineamento da Pesquisa}
\lipsum[33]

\subsection{Procedimentos para revisão da literatura}
\label{procedimentosMetd}
\lipsum[34]

\subsection{Procedimentos Metodológicos}
\lipsum[35]

\subsection{Instrumentos}
\lipsum[36]

\subsection{Análise de Dados}
\lipsum[37]

\section{Justificativa}
\lipsum[38-42]
\chapter{Resultados}\label{cap_trabalho_academico}

\lipsum[43-49]
\chapter{Discussao dos Resultados}\label{cap_trabalho_academico}

\lipsum[56-60]
% ---
% Conclusão
% ---
\chapter{Conclusão}
% ---

\lipsum[50-55] 


% ----------------------------------------------------------
% ELEMENTOS PÓS-TEXTUAIS
% ----------------------------------------------------------
\postextual
% ----------------------------------------------------------

% ----------------------------------------------------------
% Referências bibliográficas
% ----------------------------------------------------------
\bibliography{refs.bib}

% ----------------------------------------------------------
% Apêndices
% ----------------------------------------------------------
%\begin{apendicesenv}
%    \partapendices
%    \input{exemplo-apendice}
%\end{apendicesenv}
% ---

% ----------------------------------------------------------
% Anexos
% ----------------------------------------------------------
%\begin{anexosenv}
%    \partanexos
%    \input{exemplo-anexo}
%\end{anexosenv}

%---------------------------------------------------------------------
% INDICE REMISSIVO
%---------------------------------------------------------------------
\phantompart
\printindex
%---------------------------------------------------------------------

\end{document}
